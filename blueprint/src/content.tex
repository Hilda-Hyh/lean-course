% In this file you should put the actual content of the blueprint.
% It will be used both by the web and the print version.
% It should *not* include the \begin{document}
%
% If you want to split the blueprint content into several files then
% the current file can be a simple sequence of \input. Otherwise It
% can start with a \section or \chapter for instance.


\chapter{Introduction}

The goal of this project is to formalize a purely combinatorial description of curve neighborhoods in the affine flag variety of type $\widetilde{A}_1$.

The affine Weyl group of type $\widetilde{A}_1$ is the infinite dihedral group
$D_\infty$. Its Bruhat order admits a particularly simple description in terms of word length. This allows one to describe curve neighborhoods explicitly in terms of maximal elements of certain finite subsets.

The main result shows that the curve neighborhood of any element $u$ can be described as a left translate of a finite, explicitly computable set.

\chapter{The infinite dihedral group}

\begin{definition}\label{dihedral_group}
  \leanok
  Let $D_\infty$ be the group generated by two involutions $s_0, s_1$
  with relations
  \[
    s_0^2 = s_1^2 = 1.
  \]
\end{definition}

\begin{definition}\label{length}
  \leanok
  Define the length function $\ell : D_\infty \to \mathbb{N}$ as the length
  of a reduced word representing a group element.
\end{definition}

\begin{lemma}\label{length_basic}
  \uses{length}
  For all $u \in D_\infty$:
  \begin{enumerate}
    \item $\ell(u) = \ell(u^{-1})$,
    \item $\ell(uv) \le \ell(u) + \ell(v)$.
  \end{enumerate}
\end{lemma}
\begin{proof}
  \leanok
  By reduced word manipulations.
\end{proof}

\begin{lemma}\label{length_add_or_cancel}
  \uses{length}
  For all $u,v \in D_\infty$ with $\ell(u) \le \ell(v)$,
  \[
    \ell(uv) = \ell(u) + \ell(v)
    \quad \text{or} \quad
    \ell(uv) = \ell(v) - \ell(u).
  \]
\end{lemma}
\begin{proof}
  \leanok
  Let $u,v \in D_\infty$ with $\ell(u) \le \ell(v)$.

  Since $D_\infty$ is an infinite dihedral group, we can represent any element
  as a reduced word in the generators $s_0, s_1$. The product $uv$ is obtained
  by concatenating the reduced words.

  The only possible reductions occur where consecutive equal letters appear.
  Since $s_i^2 = 1$ and letters must alternate in reduced words, cancellation
  is deterministic.

  \medskip

  \noindent\textbf{Case 1: No cancellation.}
  If the last letter of $u$ differs from the first letter of $v$, the concatenated
  word is already reduced. Thus $\ell(uv) = \ell(u) + \ell(v)$.

  \medskip

  \noindent\textbf{Case 2: Maximal cancellation occurs.}
  If the last letter of $u$ equals the first letter of $v$, then cancellation
  proceeds from the concatenation point. By alternation and the constraint
  $\ell(u) \le \ell(v)$, the entire word $u$ cancels, leaving
  $\ell(uv) = \ell(v) - \ell(u)$.

  This completes the proof.
\end{proof}


\chapter{Degrees and roots}

\begin{definition}\label{degree}
  \leanok
  Define the degree of an element $w \in D_\infty$ as
  \[
    \deg(w) := (\# s_0(w), \# s_1(w)) \in \mathbb{N}^2,
  \]
  where the counts are taken in a reduced word.
\end{definition}

\begin{definition}\label{root}
  \leanok
  A degree $\alpha = (a,b) \in \mathbb{N}^2$ is called a root if
  \[
    |a-b| = 1.
  \]
\end{definition}

\begin{definition}\label{root_reflection}
  \uses{root}
  \leanok
  For each root $\alpha$, define the corresponding root reflection
  $s_\alpha \in D_\infty$.
\end{definition}

\begin{lemma}\label{odd_length_root}
  \uses{length, root_reflection}
  An element $w \in D_\infty$ is a root reflection if and only if
  $\ell(w)$ is odd.
\end{lemma}
\begin{proof}
  \leanok
  By direct inspection of reduced words.
\end{proof}

\chapter{Chains and order}

\begin{definition}\label{chain}
  \leanok
  Let $u,v \in D_\infty$ and $d \in \mathbb{N}^2$.
  A $d$-chain from $u$ to $v$ is a finite sequence of root reflections
  whose total degree is $d$ and whose product sends $u$ to $v$.
\end{definition}

\begin{definition}\label{bruhat}
  \leanok
  Define the Bruhat order on $D_\infty$ by
  \[
    u \le v \iff \text{there exists a chain from } u \text{ to } v.
  \]
\end{definition}

\begin{lemma}\label{bruhat_length}
  \uses{bruhat, length}
  For all $u,v \in D_\infty$,
  \[
    u \le v \iff \ell(u) \le \ell(v).
  \]
\end{lemma}
\begin{proof}
  \leanok
  Chains strictly increase length, and any length increase is realizable.
\end{proof}

\chapter{The sets $A_d(u)$}

\begin{definition}\label{Ad}
  \uses{degree, length}
  \leanok
  For $u \in D_\infty$ and $d \in \mathbb{N}^2$, define
  \[
    A_d(u) := \{ v \in D_\infty \mid
    \ell(uv) = \ell(u) + \ell(v),\ \deg(v) \le d \}.
  \]
\end{definition}

\begin{definition}\label{maxAd}
  \uses{Ad, bruhat}
  \leanok
  Let $\max A_d(u)$ be the set of maximal elements of $A_d(u)$
  with respect to the Bruhat order.
\end{definition}

\begin{lemma}\label{maxAd_cardinality}
  \uses{maxAd}
  If $d \ne (a,a)$ or $u \ne 1$, then $\max A_d(u)$ has exactly one element.
  Otherwise it has exactly two elements.
\end{lemma}
\begin{proof}
  \leanok
  By explicit classification of reduced words.
\end{proof}

\chapter{Curve neighborhoods}

\begin{definition}\label{curve_neighborhood}
  \leanok
  For $u \in D_\infty$ and $d \in \mathbb{N}^2$, define the curve neighborhood
  \[
    \mathcal{O}_d(u)
  \]
  as the set of Bruhat-maximal elements reachable from $u$ by chains
  of degree $\le d$.
\end{definition}

\begin{lemma}\label{curve_neighborhood_one}
  \uses{curve_neighborhood, maxAd}
  \[
    \mathcal{O}_d(1) = \max A_d(1).
  \]
\end{lemma}
\begin{proof}
  Let $v \in D_\infty$.

  By definition, $v \in \mathcal{O}_d(1)$ if and only if
  $v$ is Bruhat-maximal among elements reachable from $1$
  by chains of total degree $\le d$.

  Such a chain corresponds exactly to a reduced word for $v$,
  whose total degree equals $\deg(v)$ and whose length is $\ell(v)$.
  Hence $v$ is reachable if and only if $\deg(v) \le d$,
  i.e.\ $v \in A_d(1)$.

  Since the Bruhat order coincides with the length order
  (Lemma~\ref{bruhat_length}),
  Bruhat-maximality in $\mathcal{O}_d(1)$ is equivalent to maximality in
  $A_d(1)$.

  Therefore $\mathcal{O}_d(1) = \max A_d(1)$.
\end{proof}



\begin{lemma}\label{curve_neighborhood_translate}
  \uses{curve_neighborhood, Ad}
  For all $u,v \in D_\infty$,
  if $v \in \mathcal{O}_d(u)$ then $u^{-1}v \in A_d(u)$.
\end{lemma}
\begin{proof}
  Let $v \in \mathcal{O}_d(u)$ and set $w := u^{-1}v$.
  
  By definition of $\mathcal{O}_d(u)$, $v$ is Bruhat-maximal among elements
  reachable from $u$ by chains of total degree $\le d$. 
  
  Any chain from $u$ to $v$ remains valid after left-multiplication by $u^{-1}$,
  yielding a chain from $1$ to $w$ with the same total degree. Thus $\deg(w) \le d$.

  By Lemma~\ref{length_add_or_cancel}, we have either
  \[
    \ell(v) = \ell(u) + \ell(w) \quad \text{or} \quad \ell(v) = |\ell(w) - \ell(u)|.
  \]
  
  If $\ell(v) < \ell(u) + \ell(w)$, then there exists an element of Bruhat-greater height
  than $v$ reachable from $u$ with degree $\le d$, contradicting maximality of $v$.
  
  Hence $\ell(u \cdot w) = \ell(u) + \ell(w)$, so $w \in A_d(u)$.
\end{proof}



\chapter{Main theorem}

\begin{theorem}\label{main_theorem}
  \uses{curve_neighborhood, maxAd}
  For all $u \in D_\infty$ and all $d \ne 0$,
  \[
    \mathcal{O}_d(u) = \{ u \cdot w \mid w \in \max A_d(u) \}.
  \]
\end{theorem}

\begin{proof}
  We prove both inclusions.

  \medskip

  \noindent\textbf{Forward inclusion:} Let $v \in \mathcal{O}_d(u)$.
  By Lemma~\ref{curve_neighborhood_translate}, $w := u^{-1}v \in A_d(u)$.
  The maximality of $v$ in $\mathcal{O}_d(u)$ (with respect to the Bruhat order)
  directly implies the maximality of $w$ in $A_d(u)$.
  Thus $v = u \cdot w$ for some $w \in \max A_d(u)$.

  \medskip

  \noindent\textbf{Backward inclusion:} Let $w \in \max A_d(u)$ and set $v := u \cdot w$.
  By definition of $A_d(u)$, any element $w \in A_d(u)$ satisfies $\ell(u \cdot w) = \ell(u) + \ell(w)$.
  
  This equality means the reduced word for $w$ defines a valid chain from $u$ to $v$
  with total degree $\deg(w) \le d$.
  
  The maximality of $w$ in $A_d(u)$ implies that $v = u \cdot w$ is Bruhat-maximal
  among all elements reachable from $u$ by chains of degree $\le d$,
  hence $v \in \mathcal{O}_d(u)$.

  The theorem follows.
\end{proof}
